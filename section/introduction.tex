\section{Introduction}
Multi-agent pathfinding or MAPF \cite{ststfekomawaliatcokubabo19a,ststfekomawaliatcokubabo19b,stern19a} for short, is a fundamental AI problem that has a wide-range real application: GPS, video-games, routing, planning, traffic control etc. In few words, consider agents moving in an environment, going from an initial to a goal position; MAPF is about finding a path for each agents, such as their is no collision between them. Multiple approaches exist; Search-algorithm (CBS \cite{shstfest15a}) or reduction solving based ( \cite{barsva19a}). 
In this report we focus on an approach call ``Plan Merging'', this approach aims to solve MAPF problems by using two distinct steps; Computing a set of paths for each agents independently (which correspond to classic pathfinding / single-agent pathfinding~\cite{foghkuhagu21a}). We will refer to this step as Individual Path Finding, in short IPF. The second step, to find a solution avoiding collision using the previously computed paths.Formalization of each step will be described in their respective sections. The interest in this approach is due to pathfinding complexity which is lower than MAPF complexity \cite{nebel19a}; the idea is then to use this property to expect saving computation time or space complexity. 
The approach in its definition is close to CBS, however, the planning part of CBS stops if a conflict occurs and re-iter the planning part considering the conflict previously encountered which is not the case for the IPF; the conflict handling would be in the merging section which could be a CBS algorithm for instance.
The work achieved tries to formally define these two steps but also tries to provide workflow and different approaches with their formal definitions. The report includes a background part which formally introduce the notion of MAPF and also some definitions and notions that will be used in the whole report. The report then includes the Individual Path Finding section describes formalization of different cost or objective function for paths such as, functions based on the intrinsic property of a path, functions based on the path in the graph and functions based on other agents paths. The report then describes formalization of Plan Merging by introducing a witness solver, definitions, heuristics and approaches. And then, finally conclude. 

